\cvsection{Research Summary}
\descriptionstyle{
My research aims to uncover how the human visual system encodes information about space and time. During my PhD, I focused on four main topics: (1) perceptual asymmetries of motion in psychophysics; (2) neural responses to motion directions and the influence of contextual surround using functional MRI; (3) the functional and retinotopic organization of motion-responsive cortical areas; and (4) the temporal profile of oculomotor responses (including microsaccades and pupil) as markers of task difficulty. 
\vspace{1em} 
\\
My past and present research experience spans both academic and clinical settings, and involves a combination of psychophysics, neuroimaging, eye tracking, and computational modeling. I have taught undergraduates several topics related to visual neuroscience and programming (click {\href{https://drive.google.com/drive/folders/1E2_9Pv1tMpVRpZjle87DOJYnVqT5q4gZ?usp=sharing}{\textcolor{purple}{here}}} for sample recordings and materials). Aside from my empirical research aims and teaching, I have a keen interest in philosophy and history of science, and I am additionally committed to advancing reproducible science.  
\vspace{1em} 
\\
}