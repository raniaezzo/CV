\cvsection{Employment}
\begin{cventries}
  \cventry
    {Postdoctoral Fellow}
    {Rokers Vision Lab, New York University Abu Dhabi (NYUAD)}
    {Abu Dhabi, United Arab Emirates}
    {August. 2025 - present}
    {
      \begin{cvitems}
        \item {Focus on the early cortical stages of visual processing, with a particular emphasis on how the human brain encodes motion using a range of methods, including psychophysics, virtual reality and functional MRI.}
      \end{cvitems}
    }
  \cventry
    {Data Analyst on Imaging Research Operations Team}
    {Dickerson Lab, Massachusetts General Hospital Research Institute (MGH)}
    {Charlestown, MA}
    {Nov. 2018 - July 2020}
    {
      \begin{cvitems}
        \item {Developed automated Python pipelines for multi-site (10+) neuroimaging studies (>600 participants), including DICOM unpacking, preprocessing, QC, reporting, and database integration. Applied machine learning techniques to classify clinical phenotypes for Alzheimer's Disease, Primary Progressive Aphasia, and Posterior Cortical Atrophy.}
      \end{cvitems}
    }
  \cventry
    {Technical Associate}
    {Saxe Lab, Massachusetts Institute of Technology's McGovern Institute (MIT)}
    {Cambridge, MA}
    {Nov. 2017 - Aug. 2018}
    {
      \begin{cvitems}
        \item {Managed and processed multi-site (6+) MRI datasets (>300 participants). Performed quality control using machine learning techniques to aggregate data across aquisition sites to investigate face perception in individuals with autism.}
      \end{cvitems}
    }
  \cventry
    {Research Assistant}
    {Active Perception Lab, Boston University's Psychological \& Brain Sciences (BU)}
    {Boston, MA}
    {Jul. 2016 - Nov. 2017}
    {
      \begin{cvitems}
	\item {Collected and analyzed psychophysics data from multiple eye-trackers (C++, MATLAB); co-developed software for custom eye tracker and lead quality control efforts for various eye trackers. Focused on the role of microsaccades and ocular drift, as well as visual attention within the foveola.}
      \end{cvitems}
    }
\end{cventries}
